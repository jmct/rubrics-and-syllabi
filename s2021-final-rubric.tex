\documentclass{article}
\usepackage[utf8]{inputenc}
\usepackage{amsmath}
\usepackage{mathtools}
\usepackage{multirow}
\usepackage{hyperref}
\usepackage{xcolor}
\usepackage{tikz}
\usepackage{booktabs}
\definecolor{lucid-blue}{RGB}{0,125,179}
\definecolor{my-magenta}{RGB}{184,26,91}

\hypersetup{
    colorlinks=true,
    linkcolor=lucid-blue,
    filecolor=my-magenta,      
    urlcolor=my-magenta,
}

\newcommand{\redit}[1]{%
  {\color{my-magenta}#1}%
}

\newcommand{\redemph}[1]{%
  {\color{my-magenta}\emph{#1}}%
}

\newcommand\mydot{{\color{lucid-blue}\bullet}\mathllap{\circ}}

\title{CMSC 320: Introduction to Data Science\\
      \textbf{Final Project: A Tutorial}}
\author{Jos\'{e} Manuel Calder\'{o}n Trilla}
\date{\fbox{\fbox{\textbf{\redit{Due by 16:00 (EDT) on May 17, 2021}}}}}

\begin{document}
\maketitle

%%%%%%%%%%%%%%%%%%%%%%%%%%%%%%%%%%%%%%%%%%%%%%%%%%%%%%%%%%%%%%%%%%%%%%%%%%%%%%%%
%%%%% Motivation
%%%%%%%%%%%%%%%%%%%%%%%%%%%%%%%%%%%%%%%%%%%%%%%%%%%%%%%%%%%%%%%%%%%%%%%%%%%%%%%%

\section*{Motivation}

There will be no final exam for CMSC320, instead students are asked to submit a
\emph{tutorial} that walks the reader through the Data Science pipeline. The
subject matter of this tutorial is far less important that the ability to
communicate the approach throughout and a meaningful discussion of the
implications/interpretations of the final results.

For the purposes of this tutorial, we will assume that `The Data Science
Pipeline' has the following phases:

\begin{enumerate}
  \item Data collection/curation + parsing (if necessary)
  \item Data management/representation
  \item Exploratory data analysis
  \item Hypothesis testing and machine learning
  \item Communication of insights attained
\end{enumerate}

It is required that each tutorial is a \emph{self-contained} artifact, using a
combination of Mardown and Python code within a Jupyter Notebook. This artifact
should be \emph{publically available} on the web.
\href{https://pages.github.com/}{Github Pages} is a reasonable choice for this,
but it is not required (\href{}{Google Collab} is another off-the-shelf
solution). We will discuss this later in this document.

The dataset chosen should be publicly available (so that we can replicate your
results). Some possible sources of data:

\begin{enumerate}
  \item A curated list of datasets: \url{https://github.com/awesomedata/awesome-public-datasets}
  \item The U.S. Government is a fantastic source of open data: \url{https://www.data.gov/}
  \item Often individual States will also host their own open data, here is the link for Maryland: \url{https://opendata.maryland.gov/}
  \item Microsoft has a list of open datasets: \url{https://azure.microsoft.com/en-us/services/open-datasets/catalog/}
  \item The National Institute of Health provides many datasets, here are the COVID related datasets: \url{https://datascience.nih.gov/covid-19-open-access-resources}
\end{enumerate}

\clearpage

%%%%%%%%%%%%%%%%%%%%%%%%%%%%%%%%%%%%%%%%%%%%%%%%%%%%%%%%%%%%%%%%%%%%%%%%%%%%%%%%
%%%%% Expectations
%%%%%%%%%%%%%%%%%%%%%%%%%%%%%%%%%%%%%%%%%%%%%%%%%%%%%%%%%%%%%%%%%%%%%%%%%%%%%%%%

\section{Expectations}

In general we would expect a good submission to provide the following at a
minimum:

\begin{itemize}
  \item 1500+ words of prose in English, describing the process throughout
        and a discussion of the insights attained
  \item Approximately 150 lines of non-contrived Python
  \item Well-labelled figures showing important aspects of the analysis
  \item Links to external documentation and resources that would be useful
        in understanding the approach.
\end{itemize}

\subsection{Groups}

Groups are allowed for the final project, up to 3 people per group. As the
group size increases, so would the scale of the expectations: more people
should result in a more thorough tutorial. Groups must be pre-registered by
April 16\textsuperscript{th}.

\redit{EDIT:} There have been many questions about what constitutes `scaling'
the expectations. The graders recognize that there are some aspects of the data
science pipeline where there may only be one `right' way to do something.  For
example, if you're accessing data from a pre-populated SQL database, we do not
expect a group to discuss multiple ways of accessing the data! However, there
are many aspects in data-visualization, data-exploration, or data-analysis
(just as examples) where there \emph{are} multiple methods of accomplishing
the same task. We would expect a group to discuss more of those alternatives.
Sometimes this results in more code, but often it results in only small amount
of extra code, but a significant amount of extra discussion.

%%%%%%%%%%%%%%%%%%%%%%%%%%%%%%%%%%%%%%%%%%%%%%%%%%%%%%%%%%%%%%%%%%%%%%%%%%%%%%%%
%%%%% Examples
%%%%%%%%%%%%%%%%%%%%%%%%%%%%%%%%%%%%%%%%%%%%%%%%%%%%%%%%%%%%%%%%%%%%%%%%%%%%%%%%

\section{Examples}

The following are links to final projects from past semesters. They should be
seen as a rough guide to what is expected and to the variety of topics that can
be pursued and not as examples of the highest-scoring submissions.

\begin{itemize}
  \item \url{https://amulyavelamakanni.github.io/data-science-pipeline-tutorial/}: 
        Analysis of Freddie Mac's Single Family Loan-Level data
  \item \url{https://andrewstehman.github.io/Joe-Flacco-Is-Elite/}: 
        Investigation into whether Joe Flacco is an `elite' quarterback in the
        NFL
  \item \url{https://summerzzzy.github.io/}: 
        Analysis of global suicide rates
  \item \url{https://amygracecruz.github.io/}: 
        Attempt to predict dementia and Alzheimer's
\end{itemize}

%%%%%%%%%%%%%%%%%%%%%%%%%%%%%%%%%%%%%%%%%%%%%%%%%%%%%%%%%%%%%%%%%%%%%%%%%%%%%%%%
%%%%% Submission
%%%%%%%%%%%%%%%%%%%%%%%%%%%%%%%%%%%%%%%%%%%%%%%%%%%%%%%%%%%%%%%%%%%%%%%%%%%%%%%%

\section{The Deliverable}

We recommend `GitHub Pages' (\url{https://pages.github.com/}) for hosting your
final tutorial\footnote{It's where we host the course webpage.} Github provides
this service, free of charge, for any public repository hosted on their
platform. One of the main benefits is that GitHub does not try to be `smart'
about hosting a Jupyter Notebook, it only hosts a static HTML page. So
generating that HTML can be done independently. This also has the side benefit
that there is no risk of data sources shifting or changing, causing your
project to fail, all data manipulation, analyses, and figures are computed
ahead of time and hosted statically.

The following 3 items are required:

\begin{enumerate}
  \item A GitHub account
  \item A repository named \verb`<account-name>.github.io`, where
        \verb`<account-name>` is your unique GitHub account handle
  \item An HTML export of your final tutorial.
\end{enumerate}

You can use the same repository for storing and versioning your final project,
in fact I recommend it!), but that is not a require of GitHub pages itself.

There are many tutorials online for using GitHub pages. Personally, I have found
the official page to be sufficient: \url{https://pages.github.com/}.

\subsection{Format of your deliverable}

The formatting for the majority of the deliverable is left to your discretion.
However, each submission must begin with the title of the tutorial, providing a
rough idea of the topic, followed by your name (and all members of the group).

\subsection{Submission of your tutorial}

The submission to staff is simply the URL for your hosted tutorial. The
submission must occur by the University's specified Final Exam time:
\textbf{\textit{4:00PM EDT on May 17\textsuperscript{th} 2021}}. Due to the
short time-table for final grades, no late submissions will be accepted.

%%%%%%%%%%%%%%%%%%%%%%%%%%%%%%%%%%%%%%%%%%%%%%%%%%%%%%%%%%%%%%%%%%%%%%%%%%%%%%%%
%%%%% Assessment
%%%%%%%%%%%%%%%%%%%%%%%%%%%%%%%%%%%%%%%%%%%%%%%%%%%%%%%%%%%%%%%%%%%%%%%%%%%%%%%%

\section{Assessment}

The following dimensions of each submission will be given a rating between 1-10:

\begin{enumerate}
  \item Motivation
  \item Understanding
  \item Resources
  \item Prose
  \item Code
  \item Communication of Approach
  \item Subjective Evaluation
\end{enumerate}

\paragraph{Motivation:} each tutorial should be sufficiently motivated. If
  there is not motivation for the analysis, why would we 'do data science'
  on this topic?

\paragraph{Understanding:} the reader of the tutorial should walk away with
  \emph{some new understanding} of the topic at hand. If it's not possible
  for a reader to state `what they learned' from reading your tutorial, then
  why do the analysis?

\paragraph{Resources:} tutorials should help the reader learn a skill, but
  they should also provide a launching pad for the reader to further develop
  that skill. The tutorial should link to additional resources wherever
  appropriate, so that a well-motivated reader can read further on techniques
  that have been used in the tutorial.

\paragraph{Prose:} it's very easy to write the literal English for what the
  Python code is doing, but that's not very useful. The prose should
  \emph{enhance}, the tutorial, adding additional context and insight.

\paragraph{Code:} code should be clear and commented. Function definitions
  should be described and given context/motivation. If the prose helps
  the reader understand \emph{why} the code should  be sufficient for the
  reader to learn \emph{how}.

\paragraph{Communication of Approach:} every technical choice has alternatives,
  why did you choose the approach taken in the tutorial? A reader should walk
  away with some idea of what the trade-offs may be.

\paragraph{Subjective Evaluation:} does the tutorial seem polished and
  `publishable', or haphazard and quickly thrown together? The tutorials
  should read as well put together and having undergone a few iterations of
  editing and refinement. This should be the easiest of the dimensions.

\subsection{Grades}

Once each tutorials has been rated along each dimension, the score for
each dimension will be scaled according to the following rubric:

\vspace{\fill}

\begin{center}
\begin{tabular}{c|c}\toprule
Category                  & Points Available \\ \midrule
Motivation                & 10 \\
Understanding             & 10 \\
Resources                 & 10 \\
Prose                     & 20 \\
Code                      & 20 \\
Communication of Approach & 20 \\
Subjective Evaluation     & 10 \\ \midrule
Total Points:             & 100
\end{tabular}
\end{center}
\end{document}

